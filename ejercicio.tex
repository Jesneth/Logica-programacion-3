-Crear un programa en Javascript que realice lo siguiente:
Debe solicitar al usuario un número por prompt o por input y guardarlo.
Debe calcular el factorial del número recibido.
Debe imprimir el resultado por consola o por el DOM.
Debe ser capaz de identificar si el dato de entrada es de tipo number, en caso contrario debe mandar un mensaje de error y volver a solicitar el dato.
Prueba tu programa con las siguientes entradas para asegurarte que funcione correctamente:
Entrada: 
5
Salida:
120.
Entrada: 
6
Salida: 
720.
Pruébalo con los números que se te ocurran.
Al final de tu práctica, tienes que subir el ejercicio a tu repositorio de GitHub.
Colócalo en un repositorio llamado “logica-programacion-3”